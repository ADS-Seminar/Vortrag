\documentclass[t]{beamer}
\usetheme[height=7mm]{Rochester}
\setbeamercovered{transparent}
\setbeamertemplate{navigation symbols}{}
%\setbeamertemplate{footline}[frame number]
\setbeamertemplate{footline}
{
\leavevmode%
  \hbox{%
  \begin{beamercolorbox}[wd=\paperwidth,dp=2.5ex,ht=3ex]{}%
    \hspace*{2em}%
    {\hfill Folie \insertframenumber}%
    \hspace*{2em}%
  \end{beamercolorbox}%
  }%
}

\usepackage[german]{babel}
\usepackage[utf8]{inputenc}
\usepackage[linesnumbered,vlined]{algorithm2e}
\DontPrintSemicolon

\theoremstyle{plain}
\newtheorem{Korollar}{Korollar}


\title[Treaps]{Randomisierte Suchstrukturen: Treaps}
\author[M. Beck, R.McDaniel]{Moritz Beck, Robert McDaniel}
\date[10.06.2015]{10. Juni 2015}


\AtBeginSection[]{
    {\setbeamertemplate{footline}{}
    \begin{frame}<beamer>{Gliederung}
        \tableofcontents[currentsection]
    \end{frame}}
}
\AtBeginSubsection[]{
    {\setbeamertemplate{footline}{}
    \begin{frame}<beamer>{Gliederung}
        \tableofcontents[currentsubsection]
    \end{frame}}
}

\beamerdefaultoverlayspecification{<+->}

\begin{document}

{
\setbeamertemplate{footline}{}

\begin{frame}%[plain]
    \titlepage
\end{frame}

\begin{frame}{Gliederung}
    \tableofcontents
\end{frame}
}

\section{Einführung und Motivation}
\subsection{Grundmotivation}
\begin{frame}{Seitentitel}
    
\end{frame}
\subsection{Laufzeiten bei binären Suchbäumen}
\begin{frame}
    
\end{frame}
\subsection{Probleme bei binären Suchbäumen}
\begin{frame}
    
\end{frame}


\section{Treaps}
\subsection{Grundlagen}
\begin{frame}{Titel}
    
\end{frame}

\subsection{Existenz eindeutiger Treaps}
\begin{frame}{Titel}
    
\end{frame}

\subsection{Implementierung}
\begin{frame}{Titel}
    
\end{frame}

\subsection{Laufzeitanalyse}
\begin{frame}{Titel}
    
\end{frame}


\section*{Fazit}
\begin{frame}{Fazit}
    
\end{frame}



% \section{Skip-Lists}
% \subsection{Grundlagen}
% \begin{frame}{Titel}
    
% \end{frame}

% \section*{Ende}
% {\setbeamertemplate{footline}{}
% \begin{frame}<beamer>[c]
    % \begin{center}
    % \Huge{Fragen?}
    % \end{center}
% \end{frame}}

\end{document}